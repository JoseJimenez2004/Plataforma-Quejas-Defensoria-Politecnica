\documentclass[12pt, a4paper]{article}
\usepackage[spanish]{babel}
\usepackage[utf8]{inputenc}
\usepackage[margin=2.5cm]{geometry}
\usepackage{fancyhdr}
\usepackage{tcolorbox}
\usepackage{enumitem}
\usepackage{booktabs}
\usepackage{array}
\usepackage{listings}
\usepackage{xcolor}
\usepackage{longtable}

% Solución para el error de fancyhdr
\setlength{\headheight}{14pt}

% Configuración simplificada para listados de código
\lstset{
    basicstyle=\ttfamily\small,
    breaklines=true,
    frame=single,
    backgroundcolor=\color{gray!5},
    numbers=left,
    numberstyle=\tiny\color{gray},
}

\title{\textbf{Documento de Requerimientos Técnicos (RT)} \\
\large Sistema Integral de Gestión DDP - IPN}
\author{Dirección de Defensoría de los Derechos Politécnicos \\
Coordinación de Tecnologías de la Información}
\date{\today}

\pagestyle{fancy}
\fancyhf{}
\fancyhead[L]{\small Requerimientos Técnicos DDP}
\fancyhead[R]{\small Versión 2.0}
\fancyfoot[C]{\thepage}

\begin{document}

\maketitle

\begin{center}
    \textbf{Versión 2.0} \\
    \vspace{0.5cm}
    \textbf{Clasificación:} USO INSTITUCIONAL \\
    \textbf{Aprobado por:} Coordinador de TI DDP \\
    \textbf{Fecha de aprobación:} \today
\end{center}

\vspace{1cm}

\begin{tcolorbox}[title=Contexto Técnico Institucional, colback=gray!5!white]
    \begin{tabular}{ll}
        \textbf{Infraestructura Base:} & Red Institucional IPN \\
        \textbf{Plataforma Objetivo:} & Sistema Integral de Gestión DDP \\
        \textbf{Arquitectura:} & Híbrida (On-premise + Cloud) \\
        \textbf{Estándares:} & Normas Técnicas IPN v4.0 \\
        \textbf{Horario Operación:} & 24/7 con mantenimiento programado \\
    \end{tabular}
\end{tcolorbox}

\section*{Introducción}
Este documento especifica los requerimientos técnicos para el Sistema de Gestión DDP, estableciendo las decisiones tecnológicas, arquitectónicas y de integración necesarias para soportar los cinco procesos institucionales.

\section{Requerimientos Técnicos por Proceso}

\subsection{Proceso DDP-PO-01: Control y Gestión Administrativa de Documentos}

\begin{tcolorbox}[title=RT-01-001: Integración con Base de Datos Institucional, colback=blue!5!white]
    \begin{itemize}[leftmargin=*]
        \item \textbf{ID:} RT-01-001
        \item \textbf{Proceso:} DDP-PO-01
        \item \textbf{Categoría:} Integración
        \item \textbf{Prioridad:} Crítica
        \item \textbf{Descripción:} El registro y control de correspondencia debe realizarse mediante la base de datos institucional de la DDP.
        \item \textbf{Especificaciones Técnicas:}
        \begin{itemize}
            \item \textbf{Base de Datos:} Oracle Database 19c Enterprise Edition
            \item \textbf{Servidor:} Oracle Exadata X8M (on-premise)
            \item \textbf{Licencias:} Oracle ULA institucional
            \item \textbf{Conectividad:} JDBC Thin Client
        \end{itemize}
        \item \textbf{Esquema de Base de Datos:}
\begin{lstlisting}
-- Tabla principal de documentos
CREATE TABLE ddp_documentos (
    doc_id NUMBER PRIMARY KEY,
    numero_turno VARCHAR2(50) UNIQUE NOT NULL,
    fecha_recepcion DATE DEFAULT SYSDATE,
    remitente_id NUMBER REFERENCES ddp_remitentes(id),
    asunto VARCHAR2(500) NOT NULL,
    subdefensoria_asignada VARCHAR2(100),
    estado VARCHAR2(20) DEFAULT 'REGISTRADO',
    usuario_registro VARCHAR2(50),
    fecha_registro TIMESTAMP DEFAULT SYSTIMESTAMP
);
\end{lstlisting}
        \item \textbf{Requisitos de Conexión:}
        \begin{itemize}
            \item Connection Pool: mínimo 50 conexiones
            \item Timeout: 30 segundos
            \item Reintentos: 3 con backoff exponencial
            \item Encriptación: TLS 1.2 para conexiones
        \end{itemize}
        \item \textbf{Monitoreo:} Oracle Enterprise Manager para supervisión
        \item \textbf{Backup:} RMAN incremental diario + completo semanal
    \end{itemize}
\end{tcolorbox}

\subsection{Proceso DDP-PO-02: Atención a Solicitudes de Orientación y/o Quejas}

\begin{tcolorbox}[title=RT-02-001: Integración con Portal Web Oficial, colback=green!5!white]
    \begin{itemize}[leftmargin=*]
        \item \textbf{ID:} RT-02-001
        \item \textbf{Proceso:} DDP-PO-02
        \item \textbf{Categoría:} Interfaz Web
        \item \textbf{Prioridad:} Alta
        \item \textbf{Descripción:} Se requiere la integración con el portal oficial www.ipn.mx/defensoria/ para la recepción de quejas en línea.
        \item \textbf{Arquitectura Web:}
        \begin{itemize}
            \item \textbf{Frontend:} React 18 + TypeScript
            \item \textbf{Backend:} Node.js 18 LTS
            \item \textbf{Servidor Web:} Apache 2.4 con mod\_proxy
            \item \textbf{Balanceador:} F5 BIG-IP LTM
        \end{itemize}
        \item \textbf{Ejemplo de Componente:}
\begin{lstlisting}
import React, { useState } from 'react';

const QuejasPortal = () => {
    const [queja, setQueja] = useState({
        tipo: '',
        descripcion: '',
        anonima: false
    });

    const handleSubmit = async (e) => {
        e.preventDefault();
        // Envio seguro mediante API
        const response = await fetch('/api/quejas', {
            method: 'POST',
            headers: {
                'Content-Type': 'application/json'
            },
            body: JSON.stringify(queja)
        });
    };

    return (
        <form onSubmit={handleSubmit}>
            {/* Formulario de quejas */}
        </form>
    );
};
\end{lstlisting}
        \item \textbf{Seguridad Web:}
        \begin{enumerate}
            \item HTTPS obligatorio (TLS 1.3)
            \item WAF (Web Application Firewall)
            \item Protección contra CSRF y XSS
            \item Rate limiting: 10 solicitudes/minuto por IP
            \item CAPTCHA para formularios públicos
        \end{enumerate}
        \item \textbf{Accesibilidad:} WCAG 2.1 Nivel AA
        \item \textbf{Compatibilidad:}
        \begin{itemize}
            \item Chrome 90+, Firefox 88+, Edge 90+
            \item Safari 14+ (iOS y macOS)
            \item Resolución mínima: 320px (mobile first)
        \end{itemize}
    \end{itemize}
\end{tcolorbox}

\subsection{Proceso DDP-PO-03: Acciones de Formación en Derechos Humanos y Politécnicos}

\begin{tcolorbox}[title=RT-03-001: Gestión de Datos y Archivo, colback=purple!5!white]
    \begin{itemize}[leftmargin=*]
        \item \textbf{ID:} RT-03-001
        \item \textbf{Proceso:} DDP-PO-03
        \item \textbf{Categoría:} Gestión de Datos
        \item \textbf{Prioridad:} Media
        \item \textbf{Descripción:} El sistema debe archivar los datos de cada acción para atender reportes y solicitudes de información pública.
        \item \textbf{Arquitectura de Datos:}
        \begin{itemize}
            \item \textbf{Base Principal:} PostgreSQL 14 (datos transaccionales)
            \item \textbf{Data Warehouse:} Amazon Redshift (reportes)
            \item \textbf{Búsqueda:} Elasticsearch 8.0 (índices)
            \item \textbf{Cache:} Redis 7.0 (sesiones y datos frecuentes)
        \end{itemize}
        \item \textbf{Esquema de Formación:}
\begin{lstlisting}
-- Programa Anual de Acciones
CREATE TABLE ddp_paamdh (
    programa_id SERIAL PRIMARY KEY,
    anio INTEGER NOT NULL,
    version VARCHAR2(10),
    fecha_publicacion DATE,
    estado VARCHAR2(20) DEFAULT 'BORRADOR'
);

-- Acciones de formación
CREATE TABLE ddp_acciones_formacion (
    accion_id SERIAL PRIMARY KEY,
    programa_id INTEGER REFERENCES ddp_paamdh(programa_id),
    nombre VARCHAR2(200) NOT NULL,
    fecha_inicio DATE,
    fecha_fin DATE,
    modalidad VARCHAR2(50)
);
\end{lstlisting}
        \item \textbf{Políticas de Retención:}
        \begin{center}
        \begin{tabular}{|l|l|l|}
            \hline
            \textbf{Tipo de Dato} & \textbf{Retención} & \textbf{Destino Final} \\
            \hline
            Datos de participantes & 5 años & Archivo histórico \\
            \hline
            Materiales didácticos & 10 años & Repositorio institucional \\
            \hline
            Registros de asistencia & 7 años & Sistema de archivo \\
            \hline
            Evaluaciones & 3 años & Anonimización \\
            \hline
        \end{tabular}
        \end{center}
        \item \textbf{Backup:} Snapshots diarios + replicación cross-region
    \end{itemize}
\end{tcolorbox}

\subsection{Proceso DDP-PO-04: Acciones de Promoción en Derechos Humanos y Politécnicos}

\begin{tcolorbox}[title=RT-04-001: Sistema de Almacenamiento y Repositorio, colback=yellow!5!white]
    \begin{itemize}[leftmargin=*]
        \item \textbf{ID:} RT-04-001
        \item \textbf{Proceso:} DDP-PO-04
        \item \textbf{Categoría:} Almacenamiento
        \item \textbf{Prioridad:} Media
        \item \textbf{Descripción:} El sistema debe contar con un repositorio para el archivo de evidencia generada (fotos, programas, acuses).
        \item \textbf{Solución de Almacenamiento:}
        \begin{itemize}
            \item \textbf{Primary Storage:} NetApp AFF A800 (all-flash)
            \item \textbf{Capacity:} 100 TB inicial, escalable a 1 PB
            \item \textbf{Performance:} 1M IOPS, latencia < 1ms
            \item \textbf{Protocols:} NFS v4.1, SMB 3.1.1, S3 compatible
        \end{itemize}
        \item \textbf{Estructura de Repositorio:}
\begin{verbatim}
/repositorio_ddp/
   /actividades/
      /2024/
         /01-enero/
            /conferencia-derechos/
               /fotos/
               /listas/
               /programa/
            /taller-discriminacion/
   /evidencias/
      /fotograficas/
      /documentales/
   /backups/
\end{verbatim}
        \item \textbf{Metadatos Obligatorios:}
\begin{lstlisting}
{
  "actividad_id": "ACT-2024-001",
  "fecha": "2024-01-15",
  "tipo_evidencia": "FOTO",
  "formato": "JPEG",
  "resolucion": "4000x3000",
  "creado_por": "usuario_ddp"
}
\end{lstlisting}
        \item \textbf{Seguridad de Archivos:}
        \begin{enumerate}
            \item Encriptación AES-256 en reposo
            \item Control de acceso basado en roles
            \item Logging de acceso detallado
            \item Prevención de pérdida de datos (DLP)
            \item Detección de malware en tiempo real
        \end{enumerate}
        \item \textbf{Disponibilidad:} 99.95\% SLA
    \end{itemize}
\end{tcolorbox}

\subsection{Proceso DDP-PO-05: Gestión para la Formalización de Convenios de Colaboración}

\begin{tcolorbox}[title=RT-05-001: Sistema de Comunicación Institucional, colback=orange!5!white]
    \begin{itemize}[leftmargin=*]
        \item \textbf{ID:} RT-05-001
        \item \textbf{Proceso:} DDP-PO-05
        \item \textbf{Categoría:} Comunicación
        \item \textbf{Prioridad:} Alta
        \item \textbf{Descripción:} El intercambio de propuestas y ajustes debe realizarse exclusivamente a través del correo electrónico institucional.
        \item \textbf{Infraestructura de Correo:}
        \begin{itemize}
            \item \textbf{Plataforma:} Microsoft Exchange Online (Plan 2)
            \item \textbf{Dominio:} @ipn.mx
            \item \textbf{Capacidad:} 100 GB por buzón
            \item \textbf{Retención:} Política de 7 años
        \end{itemize}
        \item \textbf{Integración con Sistema:}
\begin{lstlisting}
public class CorreoInstitucionalService {
    public async Task EnviarPropuestaConvenio(
        PropuestaConvenio propuesta, 
        List<string> destinatarios)
    {
        // Envio mediante EWS (Exchange Web Services)
        // Registro en base de datos
    }
}
\end{lstlisting}
        \item \textbf{Características Requeridas:}
        \begin{enumerate}
            \item Cifrado de extremo a extremo (S/MIME)
            \item Firma digital de mensajes
            \item Acuse de recibo electrónico
            \item Archivado automático según política
            \item Búsqueda avanzada con eDiscovery
        \end{enumerate}
        \item \textbf{Seguridad:}
        \begin{itemize}
            \item Microsoft Defender for Office 365
            \item Anti-phishing y anti-malware
            \item Protección contra pérdida de datos
            \item Quarantine para correo sospechoso
        \end{itemize}
        \item \textbf{Disponibilidad:} 99.9\% SLA Microsoft
    \end{itemize}
\end{tcolorbox}

\section{Requerimientos Técnicos Transversales}

\begin{tcolorbox}[title=RT-T-001: Infraestructura de Seguridad, colback=red!5!white]
    \begin{itemize}[leftmargin=*]
        \item \textbf{ID:} RT-T-001
        \item \textbf{Procesos:} Todos
        \item \textbf{Categoría:} Seguridad
        \item \textbf{Prioridad:} Crítica
        \item \textbf{Descripción:} Infraestructura de seguridad integral para todos los procesos DDP.
        \item \textbf{Componentes de Seguridad:}
        \begin{center}
        \begin{tabular}{|l|l|l|}
            \hline
            \textbf{Componente} & \textbf{Tecnología} & \textbf{Responsabilidad} \\
            \hline
            Firewall & Palo Alto PA-5200 & Red perimetral \\
            \hline
            IDS/IPS & Cisco Firepower & Detección de intrusos \\
            \hline
            SIEM & Splunk Enterprise & Correlación de logs \\
            \hline
            IAM & Microsoft Azure AD & Gestión de identidades \\
            \hline
            PAM & CyberArk & Privileged Access Management \\
            \hline
            DLP & Symantec & Prevención de fuga de datos \\
            \hline
        \end{tabular}
        \end{center}
        \item \textbf{Estándares de Cumplimiento:}
        \begin{itemize}
            \item ISO 27001:2013
            \item NIST Cybersecurity Framework
            \item Ley Federal de Protección de Datos Personales
            \item Políticas de Seguridad IPN
        \end{itemize}
        \item \textbf{Auditoría:} Escaneos de vulnerabilidades mensuales
        \item \textbf{Respuesta a Incidentes:} Plan documentado y equipo dedicado
    \end{itemize}
\end{tcolorbox}

\begin{tcolorbox}[title=RT-T-002: Monitoreo y Operaciones, colback=gray!5!white]
    \begin{itemize}[leftmargin=*]
        \item \textbf{ID:} RT-T-002
        \item \textbf{Procesos:} Todos
        \item \textbf{Categoría:} Operaciones
        \item \textbf{Prioridad:} Alta
        \item \textbf{Descripción:} Sistema de monitoreo unificado para toda la infraestructura.
        \item \textbf{Stack de Monitoreo:}
        \begin{itemize}
            \item \textbf{Métricas:} Prometheus + Grafana
            \item \textbf{Logs:} ELK Stack (Elasticsearch, Logstash, Kibana)
            \item \textbf{Trazas:} Jaeger para distributed tracing
            \item \textbf{Alertas:} PagerDuty + OpsGenie
        \end{itemize}
        \item \textbf{Métricas Clave:}
\begin{lstlisting}
# Dashboard DDP-PO-01: Control Documental
- Documentos registrados por hora
- Tiempo promedio de procesamiento
- Errores de validacion
- Usuarios activos concurrentes

# Alertas criticas:
- Disponibilidad sistema < 99%
- Tiempo respuesta > 10 segundos
- Errores consecutivos > 100
- Storage utilizado > 90%
\end{lstlisting}
        \item \textbf{SLA Objetivo:} 99.5\% para todos los sistemas
        \item \textbf{MTTR:} < 4 horas para incidentes críticos
    \end{itemize}
\end{tcolorbox}

\section{Matriz de Compatibilidad Tecnológica}
\begin{longtable}{|l|l|l|l|}
    \hline
    \textbf{Componente} & \textbf{Tecnología} & \textbf{Versión} & \textbf{Procesos} \\
    \hline
    Base de Datos & Oracle Database & 19c Enterprise & DDP-PO-01, DDP-PO-03 \\
    \hline
    Servidor Web & Apache HTTP Server & 2.4.56 & DDP-PO-02 \\
    \hline
    Frontend Framework & React & 18.2.0 & DDP-PO-02 \\
    \hline
    Backend Runtime & Node.js & 18.16 LTS & DDP-PO-02, DDP-PO-04 \\
    \hline
    Storage & NetApp AFF & A800 & DDP-PO-04 \\
    \hline
    Correo & Exchange Online & Plan 2 & DDP-PO-05 \\
    \hline
    Monitoreo & Prometheus/Grafana & 2.45/9.5 & Todos \\
    \hline
    Contenedores & Docker & 24.0 & Todos \\
    \hline
    Orquestación & Kubernetes & 1.27 & Todos \\
    \hline
    Logging & Elastic Stack & 8.0 & Todos \\
    \hline
\end{longtable}

\section{Cronograma de Implementación}
\begin{table}[h]
    \centering
    \begin{tabular}{|l|l|l|l|}
        \hline
        \textbf{Fase} & \textbf{Procesos} & \textbf{Duración} & \textbf{Entrega} \\
        \hline
        Fase 1 & DDP-PO-01, DDP-PO-02 & 3 meses & Q1 2024 \\
        \hline
        Fase 2 & DDP-PO-03, DDP-PO-04 & 4 meses & Q2 2024 \\
        \hline
        Fase 3 & DDP-PO-05 & 2 meses & Q3 2024 \\
        \hline
        Fase 4 & Integración total & 3 meses & Q4 2024 \\
        \hline
        Soporte & Todos & Permanente & Post-implementación \\
        \hline
    \end{tabular}
    \caption{Cronograma de implementación técnica}
\end{table}

\section*{Firmas de Validación Técnica}
\vspace{1cm}

\begin{tabular}{p{0.3\textwidth}p{0.3\textwidth}p{0.3\textwidth}}
    \textbf{Elaborado por:} & \textbf{Revisado por:} & \textbf{Aprobado por:} \\
    \vspace{2cm} & \vspace{2cm} & \vspace{2cm} \\
    \hline
    [Nombre y firma] & [Nombre y firma] & [Nombre y firma] \\
    Arquitecto de Soluciones & Jefe de Infraestructura & Coordinador de TI DDP \\
\end{tabular}

\end{document}