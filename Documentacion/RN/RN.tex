\documentclass[12pt, a4paper]{article}
\usepackage[spanish]{babel}
\usepackage[utf8]{inputenc}
\usepackage[margin=2.5cm]{geometry}
\usepackage{fancyhdr}
\usepackage{tcolorbox}
\usepackage{enumitem}
\usepackage{booktabs}
\usepackage{array}
\usepackage{float}
\usepackage{longtable}

% Solución para el error de fancyhdr
\setlength{\headheight}{14pt}

\title{\textbf{Documento de Reglas de Negocio (RN)} \\
\large Sistema de Gestión DDP - Instituto Politécnico Nacional}
\author{Defensoría de los Derechos Politécnicos}
\date{\today}

\pagestyle{fancy}
\fancyhf{}
\fancyhead[L]{\small Reglas de Negocio DDP}
\fancyhead[R]{\small Versión 2.0}
\fancyfoot[C]{\thepage}

\begin{document}

\maketitle

\begin{center}
    \textbf{Versión 2.0} \\
    \vspace{0.5cm}
    \textbf{Vigencia:} Permanente hasta modificación \\
    \textbf{Aprobado por:} Titular de la DDP \\
    \textbf{Fecha de aprobación:} \today
\end{center}

\vspace{1cm}

\begin{tcolorbox}[title=Marco Normativo, colback=purple!5!white]
    \begin{tabular}{ll}
        \textbf{Documento Base:} & Manual de Procedimientos DDP \\
        \textbf{Normativa:} & Reglamento Interno del IPN \\
        \textbf{Versiones:} & Actualizado según reformas 2024 \\
        \textbf{Responsable:} & Departamento Jurídico DDP \\
        \textbf{Estado:} & Vigente \\
    \end{tabular}
\end{tcolorbox}

\section*{Introducción}
Este documento contiene las reglas de negocio institucionales que rigen los procesos de la Defensoría de los Derechos Politécnicos. Estas reglas son de obligatorio cumplimiento y se derivan del marco normativo del IPN.

\section{Clasificación por Procesos}
Las reglas de negocio se organizan según los procesos establecidos en el manual DDP:

\begin{table}[h]
    \centering
    \begin{tabular}{|c|l|p{3cm}|p{4cm}|}
        \hline
        \textbf{Código} & \textbf{Proceso} & \textbf{Número de RN} & \textbf{Tipo Principal} \\
        \hline
        DDP-PO-01 & Control Documental & 2 & Validación, Restricción \\
        \hline
        DDP-PO-02 & Atención de Quejas & 2 & Tiempos, Conclusión \\
        \hline
        DDP-PO-03 & Formación & 2 & Diagnóstico, Acreditación \\
        \hline
        DDP-PO-04 & Promoción & 2 & Calendarización, Evidencia \\
        \hline
        DDP-PO-05 & Convenios & 2 & Responsabilidad, Formalización \\
        \hline
    \end{tabular}
    \caption{Distribución de reglas por proceso}
\end{table}

\section{Reglas de Negocio por Proceso}

\subsection{Proceso DDP-PO-01: Control y Gestión Administrativa de Documentos}

\begin{tcolorbox}[title=RN-01-001: Registro Exclusivo de Correspondencia, colback=red!5!white]
    \begin{itemize}[leftmargin=*]
        \item \textbf{ID:} RN-01-001
        \item \textbf{Proceso:} DDP-PO-01
        \item \textbf{Categoría:} Restricción
        \item \textbf{Prioridad:} Crítica
        \item \textbf{Vigencia:} Permanente
        \item \textbf{Origen:} Manual de Procedimientos DDP, Capítulo 3, Artículo 5
        \item \textbf{Descripción:} Toda correspondencia dirigida al titular debe ser registrada y gestionada exclusivamente a través del mecanismo establecido por dicho titular.
        \item \textbf{Expresión Formal:}
        \begin{verbatim}
        Para todo documento en Documentos_Entrada,
        si documento.destinatario = "Titular DDP" entonces
            documento.mecanismo_registro = "Sistema_Oficial_DDP"
            y documento.estado diferente de "No_Registrado"
        \end{verbatim}
        \item \textbf{Excepciones:} Ninguna
        \item \textbf{Procedimiento:}
        \begin{enumerate}
            \item Recepción física en ventanilla única
            \item Digitalización inmediata
            \item Registro en sistema con folio oficial
            \item Distribución según instructivo del titular
        \end{enumerate}
        \item \textbf{Sanciones por Incumplimiento:}
        \begin{itemize}
            \item Amonestación escrita
            \item Reporte a recursos humanos
            \item Responsabilidad administrativa
        \end{itemize}
        \item \textbf{Evidencia Requerida:} Bitácora de registro con firma digital
    \end{itemize}
\end{tcolorbox}

\begin{tcolorbox}[title=RN-01-002: Requisitos Formales de Documentos, colback=red!5!white]
    \begin{itemize}[leftmargin=*]
        \item \textbf{ID:} RN-01-002
        \item \textbf{Proceso:} DDP-PO-01
        \item \textbf{Categoría:} Validación
        \item \textbf{Prioridad:} Alta
        \item \textbf{Vigencia:} Permanente
        \item \textbf{Origen:} Reglamento de Correspondencia Oficial IPN, Art. 12
        \item \textbf{Descripción:} Los documentos que ingresen deben contar obligatoriamente con la firma del remitente y el sello de la entidad emisora.
        \item \textbf{Elementos Requeridos:}
        \begin{center}
        \begin{tabular}{|l|l|l|}
            \hline
            \textbf{Elemento} & \textbf{Obligatorio} & \textbf{Validación} \\
            \hline
            Firma autógrafa & Sí & Comparación con registro \\
            \hline
            Sello institucional & Sí & Verificación de autenticidad \\
            \hline
            Fecha de emisión & Sí & Coherencia temporal \\
            \hline
            Nombre y cargo & Sí & Existencia en directorio \\
            \hline
            Folio de referencia & No & Validación si existe \\
            \hline
        \end{tabular}
        \end{center}
        \item \textbf{Proceso de Validación:}
        \begin{enumerate}
            \item Verificación visual de firma y sello
            \item Digitalización con resolución 300 DPI
            \item Registro de anomalías si las hay
            \item Comunicación al remitente en caso de deficiencias
        \end{enumerate}
        \item \textbf{Documentos Exentos:}
        \begin{itemize}
            \item Comunicaciones internas del IPN
            \item Documentos anónimos (proceso especial)
            \item Notificaciones judiciales
        \end{itemize}
        \item \textbf{Mensaje de Error:} "El documento no cumple con los requisitos formales. Favor de verificar firma y sello."
    \end{itemize}
\end{tcolorbox}

\subsection{Proceso DDP-PO-02: Atención a Solicitudes de Orientación y/o Quejas}

\begin{tcolorbox}[title=RN-02-001: Plazos de Respuesta Institucionales, colback=blue!5!white]
    \begin{itemize}[leftmargin=*]
        \item \textbf{ID:} RN-02-001
        \item \textbf{Proceso:} DDP-PO-02
        \item \textbf{Categoría:} Tiempos
        \item \textbf{Prioridad:} Alta
        \item \textbf{Vigencia:} Hasta reforma normativa
        \item \textbf{Origen:} Ley Federal de Procedimiento Administrativo, Art. 8
        \item \textbf{Descripción:} Las dependencias tienen un plazo máximo de 10 días hábiles para responder a la primera solicitud de información y 5 días para las subsecuentes.
        \item \textbf{Cálculo de Plazos:}
        \begin{verbatim}
        Fecha_Limite = Fecha_Recepcion + Dias_Habiles
        donde:
        - Dias_Habiles = 10 (primera solicitud)
        - Dias_Habiles = 5 (subsiguientes)
        - Excluye: sabados, domingos, dias inhabiles oficiales
        \end{verbatim}
        \item \textbf{Calendario Oficial:} Según calendario escolar IPN
        \item \textbf{Extensiones:} Máximo 5 días adicionales con justificación
        \item \textbf{Notificaciones:}
        \begin{itemize}
            \item Aviso a 3 días del vencimiento
            \item Recordatorio día anterior
            \item Notificación de vencimiento
            \item Reporte de incumplimiento
        \end{itemize}
        \item \textbf{Consecuencias por Incumplimiento:}
        \begin{enumerate}
            \item Reporte a superior jerárquico
            \item Cálculo de mora administrativa
            \item Responsabilidad del titular del área
        \end{enumerate}
    \end{itemize}
\end{tcolorbox}

\begin{tcolorbox}[title=RN-02-002: Conclusión Legal de Asuntos, colback=blue!5!white]
    \begin{itemize}[leftmargin=*]
        \item \textbf{ID:} RN-02-002
        \item \textbf{Proceso:} DDP-PO-02
        \item \textbf{Categoría:} Conclusión
        \item \textbf{Prioridad:} Crítica
        \item \textbf{Vigencia:} Permanente
        \item \textbf{Origen:} Reglamento Interno DDP, Capítulo 4, Artículo 15
        \item \textbf{Descripción:} Un asunto se considera legalmente concluido solo cuando se emite el acuerdo o resolución que establece la causa de conclusión y su fundamento legal.
        \item \textbf{Causas de Conclusión:}
        \begin{center}
        \begin{tabular}{|l|l|}
            \hline
            \textbf{Causa} & \textbf{Requisitos} \\
            \hline
            Satisfacción del peticionario & Acuerdo por ambas partes \\
            \hline
            Falta de competencia & Oficio de turno a instancia correspondiente \\
            \hline
            Prescripción & Cálculo de plazos legales \\
            \hline
            Desistimiento & Solicitud expresa del peticionario \\
            \hline
            Resolución fundada & Análisis jurídico completo \\
            \hline
        \end{tabular}
        \end{center}
        \item \textbf{Documentación Obligatoria:}
        \begin{enumerate}
            \item Acuerdo de conclusión firmado
            \item Fundamentación jurídica
            \item Evidencia de notificación
            \item Archivo en expediente electrónico
        \end{enumerate}
        \item \textbf{Prohibiciones:}
        \begin{itemize}
            \item No se pueden archivar asuntos sin resolución
            \item No se puede concluir sin notificación
            \item No se pueden eliminar expedientes concluidos
        \end{itemize}
        \item \textbf{Periodo de Retención:} 10 años después de conclusión
    \end{itemize}
\end{tcolorbox}

\subsection{Proceso DDP-PO-03: Acciones de Formación en Derechos Humanos y Politécnicos}

\begin{tcolorbox}[title=RN-03-001: Diagnóstico Previo de Necesidades, colback=green!5!white]
    \begin{itemize}[leftmargin=*]
        \item \textbf{ID:} RN-03-001
        \item \textbf{Proceso:} DDP-PO-03
        \item \textbf{Categoría:} Diagnóstico
        \item \textbf{Prioridad:} Media
        \item \textbf{Vigencia:} Anual
        \item \textbf{Origen:} Programa Institucional de Derechos Humanos
        \item \textbf{Descripción:} La DDP debe realizar un diagnóstico previo de necesidades y demandas de la comunidad antes de elaborar el programa anual.
        \item \textbf{Métodos de Diagnóstico:}
        \begin{itemize}
            \item Encuestas a comunidad politécnica
            \item Análisis de quejas recibidas
            \item Grupos focales por unidad académica
            \item Estadísticas de años anteriores
        \end{itemize}
        \item \textbf{Población Muestra:} Mínimo 10\% por segmento:
        \begin{center}
        \begin{tabular}{|l|l|l|}
            \hline
            \textbf{Segmento} & \textbf{Muestra Mínima} & \textbf{Periodicidad} \\
            \hline
            Estudiantes & 5,000 & Anual \\
            \hline
            Académicos & 500 & Anual \\
            \hline
            Administrativos & 300 & Anual \\
            \hline
            Directivos & 50 & Semestral \\
            \hline
        \end{tabular}
        \end{center}
        \item \textbf{Plazo para Diagnóstico:} Octubre-Noviembre de cada año
        \item \textbf{Validación:} Comité de Planeación DDP
        \item \textbf{Producto:} Informe de diagnóstico con propuestas
    \end{itemize}
\end{tcolorbox}

\begin{tcolorbox}[title=RN-03-002: Acreditación de Convenios de Formación, colback=green!5!white]
    \begin{itemize}[leftmargin=*]
        \item \textbf{ID:} RN-03-002
        \item \textbf{Proceso:} DDP-PO-03
        \item \textbf{Categoría:} Acreditación
        \item \textbf{Prioridad:} Alta
        \item \textbf{Vigencia:} Vigencia del convenio
        \item \textbf{Origen:} Acuerdos Interinstitucionales IPN
        \item \textbf{Descripción:} Las instituciones especializadas (CNDH, CDHCM, CONAPRED) deben aprobar y acreditar los convenios específicos de formación.
        \item \textbf{Instituciones Acreditadoras:}
        \begin{itemize}
            \item Comisión Nacional de Derechos Humanos (CNDH)
            \item Comisión de Derechos Humanos de la Ciudad de México (CDHCM)
            \item Consejo Nacional para Prevenir la Discriminación (CONAPRED)
        \end{itemize}
        \item \textbf{Proceso de Acreditación:}
        \begin{enumerate}
            \item Elaboración de propuesta por DDP
            \item Revisión por área jurídica IPN
            \item Envío a institución acreditadora
            \item Negociación de términos
            \item Firma de convenio específico
            \item Registro en sistema de convenios IPN
        \end{enumerate}
        \item \textbf{Vigencia de Acreditación:} Máximo 3 años, renovable
        \item \textbf{Documentación:}
        \begin{itemize}
            \item Convenio marco (existente)
            \item Convenio específico por programa
            \item Plan de estudios aprobado
            \item Perfiles de instructores
        \end{itemize}
    \end{itemize}
\end{tcolorbox}

\subsection{Proceso DDP-PO-04: Acciones de Promoción en Derechos Humanos y Politécnicos}

\begin{tcolorbox}[title=RN-04-001: Calendarización Escolar, colback=yellow!5!white]
    \begin{itemize}[leftmargin=*]
        \item \textbf{ID:} RN-04-001
        \item \textbf{Proceso:} DDP-PO-04
        \item \textbf{Categoría:} Calendarización
        \item \textbf{Prioridad:} Alta
        \item \textbf{Vigencia:} Anual
        \item \textbf{Origen:} Calendario Escolar Oficial IPN
        \item \textbf{Descripción:} El Plan de Trabajo Anual debe elaborarse estrictamente bajo el calendario escolar autorizado por el IPN.
        \item \textbf{Restricciones de Calendario:}
        \begin{center}
        \begin{tabular}{|l|l|}
            \hline
            \textbf{Periodo} & \textbf{Restricciones} \\
            \hline
            Inicio de semestre & Primera semana, solo actividades de inducción \\
            \hline
            Exámenes parciales & No programar actividades extracurriculares \\
            \hline
            Exámenes finales & Prohibida cualquier actividad promocional \\
            \hline
            Vacaciones & Solo actividades previa autorización especial \\
            \hline
            Días inhábiles & No programar actividades oficiales \\
            \hline
        \end{tabular}
        \end{center}
        \item \textbf{Fuente Oficial:} Calendario escolar publicado por DGIP
        \item \textbf{Actualizaciones:} Sincronización automática con calendario oficial
        \item \textbf{Validaciones:}
        \begin{enumerate}
            \item Verificación contra calendario escolar
            \item Confirmación de disponibilidad de espacios
            \item Aprobación de unidad académica anfitriona
            \item Registro en sistema de calendarización IPN
        \end{enumerate}
        \item \textbf{Sanciones por Incumplimiento:} Cancelación de actividad y reporte
    \end{itemize}
\end{tcolorbox}

\begin{tcolorbox}[title=RN-04-002: Evidencia de Actividades, colback=yellow!5!white]
    \begin{itemize}[leftmargin=*]
        \item \textbf{ID:} RN-04-002
        \item \textbf{Proceso:} DDP-PO-04
        \item \textbf{Categoría:} Evidencia
        \item \textbf{Prioridad:} Media
        \item \textbf{Vigencia:} Permanente
        \item \textbf{Origen:} Manual de Evidencia DDP, Capítulo 2
        \item \textbf{Descripción:} Se debe generar y resguardar evidencia fotográfica y listas de registro de asistencia como comprobación de cada actividad desarrollada.
        \item \textbf{Tipos de Evidencia Requerida:}
        \begin{center}
        \begin{tabular}{|l|l|l|}
            \hline
            \textbf{Tipo} & \textbf{Formato} & \textbf{Cantidad Mínima} \\
            \hline
            Fotografías & JPG, 300 DPI & 10 por actividad \\
            \hline
            Lista de asistencia & PDF firmado & 1 original + copia \\
            \hline
            Programa de actividad & PDF & 1 \\
            \hline
            Materiales utilizados & Varios & Todos los empleados \\
            \hline
            Evaluación de participantes & Sistema en línea & 70\% de asistentes \\
            \hline
        \end{tabular}
        \end{center}
        \item \textbf{Metadatos Obligatorios:}
        \begin{itemize}
            \item Fecha y hora de actividad
            \item Lugar específico
            \item Nombre de responsable
            \item Número de participantes
            \item Temática desarrollada
        \end{itemize}
        \item \textbf{Periodo de Retención:} 5 años
        \item \textbf{Respaldo:} En sistema institucional y físico en archivo
    \end{itemize}
\end{tcolorbox}

\subsection{Proceso DDP-PO-05: Gestión para la Formalización de Convenios de Colaboración}

\begin{tcolorbox}[title=RN-05-001: Responsabilidad Única de Documentación, colback=orange!5!white]
    \begin{itemize}[leftmargin=*]
        \item \textbf{ID:} RN-05-001
        \item \textbf{Proceso:} DDP-PO-05
        \item \textbf{Categoría:} Responsabilidad
        \item \textbf{Prioridad:} Alta
        \item \textbf{Vigencia:} Permanente
        \item \textbf{Origen:} Manual de Convenios IPN, Artículo 8
        \item \textbf{Descripción:} El Departamento de Promoción (DPFDH) es el único encargado de recabar la documentación probatoria requerida para formalizar alianzas.
        \item \textbf{Documentación Probatoria:}
        \begin{enumerate}
            \item Acta constitutiva de institución externa
            \item Identificación oficial del representante
            \item Poderes vigentes del firmante
            \item RFC y situación fiscal
            \item Propuesta técnica y económica
            \item Aval jurídico IPN
        \end{enumerate}
        \item \textbf{Flujo Documental:}
        \begin{verbatim}
        Institucion Externa -> DPFDH -> Area Juridica -> 
        Relaciones Internacionales -> Titular IPN
        \end{verbatim}
        \item \textbf{Plazos:}
        \begin{itemize}
            \item Recopilación: 15 días hábiles
            \item Revisión jurídica: 10 días hábiles
            \item Negociación: variable según complejidad
            \item Firma: según agenda de autoridades
        \end{itemize}
        \item \textbf{Prohibiciones:}
        \begin{itemize}
            \item Otras áreas no pueden solicitar documentación
            \item No se acepta documentación incompleta
            \item No se procesan convenios sin aval jurídico
        \end{itemize}
    \end{itemize}
\end{tcolorbox}

\begin{tcolorbox}[title=RN-05-002: Formalización de Convenios, colback=orange!5!white]
    \begin{itemize}[leftmargin=*]
        \item \textbf{ID:} RN-05-002
        \item \textbf{Proceso:} DDP-PO-05
        \item \textbf{Categoría:} Formalización
        \item \textbf{Prioridad:} Crítica
        \item \textbf{Vigencia:} Permanente
        \item \textbf{Origen:} Reglamento de Formalización de Convenios IPN
        \item \textbf{Descripción:} Al finalizar la aceptación, se deben imprimir obligatoriamente 3 tantos en original para recabar las firmas correspondientes.
        \item \textbf{Especificaciones de Impresión:}
        \begin{center}
        \begin{tabular}{|l|l|}
            \hline
            \textbf{Característica} & \textbf{Especificación} \\
            \hline
            Cantidad de originales & 3 ejemplares idénticos \\
            \hline
            Papel & Bond institucional 75 gr \\
            \hline
            Formato & Oficio (215.9 x 330.2 mm) \\
            \hline
            Numeración & Página X de Y en cada página \\
            \hline
            Firmas & Originales en todas las copias \\
            \hline
            Sellos & Húmedos en todas las firmas \\
            \hline
        \end{tabular}
        \end{center}
        \item \textbf{Distribución de Originales:}
        \begin{enumerate}
            \item Original 1: Archivo DDP
            \item Original 2: Institución externa
            \item Original 3: Archivo General IPN
        \end{enumerate}
        \item \textbf{Orden de Firmas:}
        \begin{enumerate}
            \item Representante externo
            \item Titular DDP
            \item Dirección de Relaciones Internacionales
            \item Titular IPN (según monto y alcance)
        \end{enumerate}
        \item \textbf{Registro:} Inmediato en Sistema de Convenios IPN
    \end{itemize}
\end{tcolorbox}

\section{Matriz de Trazabilidad de Reglas}
\begin{longtable}{|c|l|l|l|l|}
    \hline
    \textbf{ID RN} & \textbf{Proceso} & \textbf{Descripción} & \textbf{Prioridad} & \textbf{Vigencia} \\
    \hline
    RN-01-001 & DDP-PO-01 & Registro exclusivo de correspondencia & Crítica & Permanente \\
    \hline
    RN-01-002 & DDP-PO-01 & Requisitos formales de documentos & Alta & Permanente \\
    \hline
    RN-02-001 & DDP-PO-02 & Plazos de respuesta institucionales & Alta & Hasta reforma \\
    \hline
    RN-02-002 & DDP-PO-02 & Conclusión legal de asuntos & Crítica & Permanente \\
    \hline
    RN-03-001 & DDP-PO-03 & Diagnóstico previo de necesidades & Media & Anual \\
    \hline
    RN-03-002 & DDP-PO-03 & Acreditación de convenios de formación & Alta & Vigencia convenio \\
    \hline
    RN-04-001 & DDP-PO-04 & Calendarización escolar & Alta & Anual \\
    \hline
    RN-04-002 & DDP-PO-04 & Evidencia de actividades & Media & Permanente \\
    \hline
    RN-05-001 & DDP-PO-05 & Responsabilidad única de documentación & Alta & Permanente \\
    \hline
    RN-05-002 & DDP-PO-05 & Formalización de convenios & Crítica & Permanente \\
    \hline
\end{longtable}

\section{Proceso de Modificación de Reglas}
Las modificaciones a las reglas de negocio deben seguir estrictamente:
\begin{enumerate}
    \item Solicitud formal con justificación
    \item Análisis de impacto por área jurídica
    \item Consulta a áreas involucradas
    \item Aprobación por Comité de Reglas DDP
    \item Publicación en gaceta institucional
    \item Capacitación a personal afectado
    \item Actualización de sistemas
\end{enumerate}

\section*{Firmas de Validación}
\vspace{1cm}

\begin{tabular}{p{0.3\textwidth}p{0.3\textwidth}p{0.3\textwidth}}
    \textbf{Elaborado por:} & \textbf{Revisado por:} & \textbf{Aprobado por:} \\
    \vspace{2cm} & \vspace{2cm} & \vspace{2cm} \\
    \hline
    [Nombre y firma] & [Nombre y firma] & [Nombre y firma] \\
    Departamento Jurídico DDP & Subdefensor Administrativo & Titular de la DDP \\
\end{tabular}

\end{document}